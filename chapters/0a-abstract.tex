\chapter*{Abstract}
Discovering the appropriate applicants who fit with the job description is imperative for the organizations' success and growth. As people's personality is an accepted indicator of job performance, extracting these behavioral traits in the screening process will aid the hiring managers making sound employment decisions. Personality is a complex construct and researchers have for decades attempted to establish taxonomies for distinguishing, ordering, and naming individual differences in people's behavior and experience. One of the most accepted personality theories is the Big Five personality model, also known as the Five-Factor model. These five dimensions (openness, conscientiousness, extraversion, agreeableness, neuroticism) outline the structure of human personality. \\

With the proliferation of social media, people share their ideas, feelings, and opinions in form of videos instead of pure text. Hence, multimodal sentiment analysis (MSA) has emerged as a new research field. MSA utilizes Deep Learning (DL) at several stages including multimodal feature extraction, fusion and emotion recognition in multimodal data, including acoustic, textual, and visual channels. MSA can assist recruiters during video interviews to monitor the applicant's expressed emotions, as researchers have demonstrated that there is a strong link between personality traits and emotions. \\

The purpose of this Master's thesis research is to investigate how MSA can assist predicting personality traits based on the Big Five model in asynchronous video interviews (AVIs). A multimodal feature extraction pipeline is developed to extract audio, visual, and textual features in a video. A private AVI dataset is collected from participants within the informatics domain. DL models are used for estimating the emotion distribution in interviews and a personality likelihood approach is proposed to predict behavioral traits in candidates. \\

Results show that the personality likelihood model achieves 66.7\% accuracy in prediction the three strongest traits. The use of Big Five personality test shows a unique and novel way of validating the personality prediction model on emotions extracted from candidates. MSA shows promising potential and can be used for personality prediction. Larger, comparative studies need to be performed before this method can be used by recruiters.  


