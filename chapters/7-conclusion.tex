\chapter{Conclusion}

\section{Future work}
In the research area, datasets are either annotated with emotion or personality labels. Given the discrepancy of datasets with both aspects labeled, there is a need in future work to create large-scale datasets which encompasses both emotions and corresponding personality traits. This is imperative in order to establish the ground truth data for emotion and personality detection. However, creating such emotion-personality dataset is a challenging task. It can be costly and time consuming since it is required that each sample is manually labeled by human annotators. In addition, including human workforce is a threat to the validity as the dataset is prone to error due to the annotators' subjectivity. 
