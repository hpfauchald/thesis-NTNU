\chapter*{Sammendrag}
Å oppdage de riktig jobbsøkerne som passer med stillingsbeskrivelsen er avgjør-ende for organisasjonene sin suksess og vekst. Siden kandidater sin personlighet er en akseptert indikator på jobbprestasjon, jobbtilfredshet og ansettelsesintensjon, vil det å trekke ut disse adferdstrekkene i screeningprosessen hjelpe rekrutterere til å ta fornuftige ansettelsesbeslutninger. Konstruksjonen av personalighet er kompleks og forskere har i flere tiår forsøkt å etablere taksonomier for å skille og navngi individuelle forskjeller i personers oppførsel og erfaring. En av de mest aksepterte personlighetsteoriene er Big Five personlighetsmodellen, også kjent som Femfaktormodellen. Disse fem dimensjonene (åpenhet, planmessighet, ekstroversjon, medmenneskelighet, nevrotisisme) utgjør strukturen av personalighet. Med den store økningen av sosiale medier deler personer ideer, følelser og meninger i form av videoer i stedet for ren tekst. Derfor har multimodal sentiment analysis (MSA) blitt et populært forskningsfelt. MSA benytter Deep learning i på flere stadier som for eksempel multimodal feature extraction, fusion emotion recognition i multimodal data. MSA kan hjelpe rekrutterere i et video intervju med å observere uttrykte følelser fra kandidater. Dette er verdifullt siden forskere har demonstrert at det er en sterk forbindelse mellom personalighetstrekk og følelser. Hensikten med denne studien er å undersøke hvordan MSA kan benyttes til å predikere personalighetstrekk baser på Femfaktormodellen i asynkrone video intervju (AVI). En multimodal feature extraction pipeline er utviklet for å hente ut lyd, bilde, og tekstinformasjon i en video. Et privat AVI dataset er samlet inn fra deltakere innenfor informatikk domenet. Deep learning modeller er brukt for å estimere følelsesfordelingen i intervjuer if en personlighet-prediksjonsmodell er laget for å forutsi personlighetstrekkene hos kandidater. Resultatene viser at personlighet-prediksjonsmodellen oppnår 67\% nøyaktighet i å predikere de tre sterkeste trekkene. Bruken av Big Five personlighetstest viser en unik og ny måte å validere prediksjonsmodellen. MSA viser et godt potensiale og kan bli brukt for å forutsi personalighetstrekk. I fremtiden trengs større, komperative studier før denne metoden kan benyttes av rekrutterere. 



