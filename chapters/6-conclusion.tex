\chapter{Conclusion and Future Work}
\label{chap:conclusion}
This chapter concludes the study. Section \ref{sec:conclusion} presents the conclusion, whereas Section \ref{sec:future_work} describes future research directions.  

\section{Conclusion}
\label{sec:conclusion}
\begin{comment}
    As the labour is highly competitive, it is important for recruiters to select the most appropriate candidates with speed and accuracy. Asynchronous video interview (AVI) is the major component of the screening process. Recruiters aim to capture applicants' expressed emotions in the AVIs. However, the main limitation of this method originates from the underlying uncertainty of the human recognition of emotions. The aim of this thesis was to investigate how multimodal sentiment analysis (MSA) can assist in predicting the personality traits of candidates from the emotions expressed in AVIs. Deep learning state-of-the-art COGMEN architecture was utilized to detect emotions expressed in videos. Experiments were performed on MELD, CMU-MOSEI, and IEMOCAP, three benchmarks datasets in MSA. A novel personality likelihood model was proposed in the study. This prediction model was based on the emotion-personality relationship from previous works. The model uses the emotion distribution. \\
\end{comment}

This thesis aimed to investigate how multimodal sentiment analysis (MSA) could assist in predicting personality traits from the Big Five taxonomy based on the emotions expressed in asynchronous video interviews (AVIs). A feature extraction pipeline is developed in order to extract useful audio, visual, and textual information in a video. Deep learning state-of-the-art MSA model COGMEN is used for detecting emotions from applicants. Experiments are performed on MELD, CMU-MOSEI, and IEMOCAP, three benchmark MSA datasets, for performance evaluation. Given the lack of datasets concerning an AVI setting, a new dataset named the AVI dataset is proposed. This dataset was acquired by conducting an AVI inviting participants from the informatics domain to answer four interview questions. The AVI design was carefully created considering the applicants' behavior before, after, and during the interview as well as their impact on the psychometric properties of the interview. The study also proposed a personality likelihood model to predict the three strongest traits of the candidates. Responses from the AVI are analyzed for emotion recognition and the personality prediction model predicts behavioral traits according to the emotion distribution. The thesis also presents a unique way of validating the proposed model's performance via personality self-assessment from participants. COGMEN's emotion distribution was further validated by comparing it to two MSA baseline models. \\

The findings showed that IEMOCAP was the most appropriate dataset for training in an AVI setting due to its performance and emotions labels. In the AVIs, sad, frustrated, neutral, and excited were the dominant expressed emotions, whereas happy and angry were hardly elicited. This may be due to the AVI design. In addition, it can be easier to express negative emotions discussing challenging occasions and the tendency to express neutral emotions discussing accomplishments. Further, results from the behavioral prediction showed that the personality prediction model was able to predict the three strongest traits in candidates with high accuracy. It was challenging to link distinct emotions to each dimension of the Big Five model. However, coarse-grained emotions (sentiment level) were useful in prediction the traits. Openness is correlated with a mix of positive, negative, and neutral emotions. Conscientiousness is prevalent when there is a higher degree of neutral emotions present in an interview. Extraversion is solely correlated positive emotions. Agreeableness is related to positive emotions as well as increasing in line with conscientiousness due to their overlapping characteristics. Neuroticism is associated with negative emotions. \\

The work conducted in this thesis suggests that MSA possesses a huge potential in predicting behavioral traits in job candidates. For that reason, including Artificial Intelligence capabilities in the AVIs will help recruiters select candidates with speed an accuracy. 

\section{Future work}
\label{sec:future_work}
MSA is yet in its infancy but shows it is a powerful tool within the human resources domain. Therefore, future research can performed in the following directions:
\begin{itemize}
    \item \textbf{AVI dataset}: In the research area, datasets are either annotated with emotion or personality labels. Given the discrepancy of datasets with both aspects labeled, there is a need in future work to create large-scale datasets which encompasses both emotions and corresponding personality traits. This is imperative in order to establish the ground truth data for emotion and personality detection. However, creating such emotion-personality dataset is a challenging task. It can be costly and time consuming since it is required that each sample is manually labeled by human annotators. In addition, including human work is a threat to the validity as the dataset is prone to error due to the annotators' subjectivity. \\ 
    \item \textbf{Deep learning models}: In this thesis, Deep learning (DL) models are utilized for emotion recognition. Once a comprehensive dataset is established including labels for emotions and traits, it paves the way to develop DL architectures that perform multi-class classification for both emotion and personality detection. This allows for emotions and personality traits to be detected simultaneously because they are complementary tasks. \\
    \item \textbf{Domain adaption}: A common issue with MSA is that it is domain-sensitive. Modes of the expressed emotion may vary across domains. In the future it would be interesting to develop MSA and DL personality prediction models utilizing domain adaption by learning features from an unknown domain. In this way, AVIs can be conducted in various working domains such as informatics, economics, finance, healthcare, teaching, etc. \\
    \item  \textbf{Screening platform}: While AVIs is a great activity to assess the candidates' personality, it is important to complement it with other screening activities including resume screening, reference checks, and job simulations. As a result, it would be beneficial to create a screening platform for an organization where these activities are included to ensure a comprehensive evaluation of applicants. 
\end{itemize}



